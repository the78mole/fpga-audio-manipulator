%% Artikelvorlage unter Verwendung der Artikelklasse des KOMA-Script %%
%% Basierend auf einer TeXNicCenter-Vorlage von Mark M�ller          %%
%%%%%%%%%%%%%%%%%%%%%%%%%%%%%%%%%%%%%%%%%%%%%%%%%%%%%%%%%%%%%%%%%%%%%%%

% W�hlen Sie die Optionen aus, indem Sie % vor der Option entfernen  
% Dokumentation des KOMA-Script-Packets: scrguide

%%%%%%%%%%%%%%%%%%%%%%%%%%%%%%%%%%%%%%%%%%%%%%%%%%%%%%%%%%%%%%%%%%%%%%%
%% Optionen zum Layout des Artikels                                  %%
%%%%%%%%%%%%%%%%%%%%%%%%%%%%%%%%%%%%%%%%%%%%%%%%%%%%%%%%%%%%%%%%%%%%%%%
\documentclass[%
%a5paper,							% alle weiteren Papierformat einstellbar
%landscape,						% Querformat
%10pt,								% Schriftgr��e (12pt, 11pt (Standard))
%BCOR1cm,							% Bindekorrektur, bspw. 1 cm
%DIVcalc,							% f�hrt die Satzspiegelberechnung neu aus
%											  s. scrguide 2.4
%twoside,							% Doppelseiten
%twocolumn,						% zweispaltiger Satz
%halfparskip*,				% Absatzformatierung s. scrguide 3.1
%headsepline,					% Trennline zum Seitenkopf	
%footsepline,					% Trennline zum Seitenfu�
%titlepage,						% Titelei auf eigener Seite
%normalheadings,			% �berschriften etwas kleiner (smallheadings)
%idxtotoc,						% Index im Inhaltsverzeichnis
%liststotoc,					% Abb.- und Tab.verzeichnis im Inhalt
%bibtotoc,						% Literaturverzeichnis im Inhalt
%abstracton,					% �berschrift �ber der Zusammenfassung an	
%leqno,   						% Nummerierung von Gleichungen links
%fleqn,								% Ausgabe von Gleichungen linksb�ndig
%draft								% �berlangen Zeilen in Ausgabe gekennzeichnet
]
{scrartcl}

%\pagestyle{empty}		% keine Kopf und Fu�zeile (k. Seitenzahl)
%\pagestyle{headings}	% lebender Kolumnentitel  

\usepackage{ae}        % Benutzen Sie nur
\usepackage{graphicx} %%Grafiken in pdfLaTeX

%% deutsche Anpassung %%%%%%%%%%%%%%%%%%%%%%%%%%%%%%%%%%%%%%%%%%%%%%%%%
\usepackage[ngerman]{babel}		
\usepackage[T1]{fontenc}							
\usepackage[latin1]{inputenc}		

%% Bibliographiestil %%%%%%%%%%%%%%%%%%%%%%%%%%%%%%%%%%%%%%%%%%%%%%%%%%
%\usepackage{natbib}

\begin{document}
%% Dateiendungen f�r Grafiken %%%%%%%%%%%%%%%%%%%%%%%%%%%%%%%
%% ==> Sie k�nnen hiermit die Dateiendung einer Grafik weglassen.
%% ==> Aus "\includegraphics{titel.eps}" wird "\includegraphics{titel}".
%% ==> Wenn Sie nunmehr 2 inhaltsgleiche Grafiken "titel.eps" und
%% ==> "titel.pdf" erstellen, wird jeweils nur die Grafik eingebunden,
%% ==> die von ihrem Compiler verarbeitet werden kann.
%% ==> pdfLaTeX benutzt "titel.pdf". LaTeX benutzt "titel.eps".

\pagestyle{empty} %%Keine Kopf-/Fusszeilen auf den ersten Seiten.

%%%%%%%%%%%%%%%%%%%%%%%%%%%%%%%%%%%%%%%%%%%%%%%%%%%%%%%%%%%%%%%%%%%%%%%
%% Ihr Artikel                                                       %%
%%%%%%%%%%%%%%%%%%%%%%%%%%%%%%%%%%%%%%%%%%%%%%%%%%%%%%%%%%%%%%%%%%%%%%%

%% eigene Titelseitengestaltung %%%%%%%%%%%%%%%%%%%%%%%%%%%%%%%%%%%%%%%    
%\begin{titlepage}
%Einsetzen der TXC Vorlage "Deckblatt" m�glich
%\end{titlepage}

%% Angaben zur Standardformatierung des Titels %%%%%%%%%%%%%%%%%%%%%%%%
%\titlehead{Titelkopf }
%\subject{Typisierung}
\title{Praktikum zum Test integrierter Schltungen}
\author{Daniel Glaser}
%\and{Der Name des Co-Autoren}
%\thanks{Fu�note}			% entspr. \footnote im Flie�text
%\date{}							% falls anderes, als das aktuelle gew�nscht
\publishers{Universit�t Erlangen, Lehrstuhl f�r Rechnergest�tzten Schaltungsentwurf \\ Prof. Dr.-Ing. K. Helmreich \\ Dipl.-Ing K. Schneider}

%% Widmungsseite %%%%%%%%%%%%%%%%%%%%%%%%%%%%%%%%%%%%%%%%%%%%%%%%%%%%%%
%\dedication{Widmung}

\maketitle 						% Titelei wird erzeugt

\clearpage
%% Zusammenfassung nach Titel, vor Inhaltsverzeichnis %%%%%%%%%%%%%%%%%
%\begin{abstract}
% F�r eine kurze Zusammenfassung des folgenden Artikels.
% F�r die �berschrift s. \documentclass[abstracton].
%\end{abstract}

%% Erzeugung von Verzeichnissen %%%%%%%%%%%%%%%%%%%%%%%%%%%%%%%%%%%%%%%
\tableofcontents			% Inhaltsverzeichnis
%\listoftables				% Tabellenverzeichnis
%\listoffigures				% Abbildungsverzeichnis


%% Der Text %%%%%%%%%%%%%%%%%%%%%%%%%%%%%%%%%%%%%%%%%%%%%%%%%%%%%%%%%%%

\part{Einleitung}

\section{Motivation}

Im Rahmen eines Praktikums soll den Studierenden anhand eines Analog-Digital-Umsetzers das praktische Wissen zum Test integrierten Schaltungen vermittelt werden. Da die ausgelieferten AD-Wandler bereits getestet sind und nur mit geringen Abweichungen der Kennlinie zu rechnen ist, muss eine Schaltung entwickelt werden, welche die digitalen Werte so verf�lscht, dass diese einem fehlerbehafteten AD-Wandler �hnlich sind. Folgende Fehler sollen dabei modelliert werden:

\begin{itemize}
	\item Kennlinienfehler
	\item (starkes) Rauschen
	\item Quantisierungsfehler ("`missing Codes"', falsche Codes)
\end{itemize}

Die eingef�gten Fehler sollen in Ihrer Wirkung einstellbar sein. Weiterhin war gew�nscht, eine Audio-Signalquelle und einen Lautsprecher anschlie�en zu k�nnen, um die Effekte direkt h�rbar zu machen. Auch eine Messung des unverf�lschten Signals in einem Tester f�r integrierte Schaltkreise ist m�glich.

\part{Konzept}

\section{Grundlegende �berlegungen}

Die gew�nschte Quelle und Senke erfordern eine Anpassung der Schaltung sowohl im Ein-, als auch im Ausgangspfad. Ist eine Leistungsverst�rkung am Ausgang gew�nscht, kommt ein einfacher NF-Kleinleistungsverst�rker zum Einsatz, die bereits in vielen Baus�tzen als Referenzdesign vorliegen. Schwieriger wird allrdings die Verf�lschung der digitalen Daten in Echtzeit. Dazu soll ein FPGA zum Einsatz kommen, der anhand mehrerer Module die gew�nschte Ver�nderung durchf�hrt. M�gliche Varianten der Implementierung sind folgende:

\begin{itemize}
	\item Look-Up Tables
	\begin{itemize}
		\item Mit Interpolation
		\item Ohne Interpolation\footnote{Kommt aufgrund ungen�gender Ressourcen nicht zum Einsatz}
	\end{itemize}
	\item Einfache Arithmetik
	\begin{itemize}
		\item Addition
		\item Multiplikation
		\item Potenzen niedriger Ordnung
	\end{itemize}
	\item Komplexe Berechnungen (CORDIC)
	\item Pseudo-Random Noise
\end{itemize}



%% Bibliographie unter Verwendung von dinnat %%%%%%%%%%%%%%%%%%%%%%%%%%
%\setbibpreamble{Pr�ambel}		% Text vor dem Verzeichnis
%\bibliographystyle{dinat}
%\bibliography{bibliographie}	% Sie ben�tigen einen *.bib-Datei

\end{document}
